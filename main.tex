%%%%%%%%%%%%%%%%%%%%%%%%%%%%%%%%%%%%%%%%%
% Beamer Presentation
% LaTeX Template
% Version 2.0 (March 8, 2022)
%
% This template originates from:
% https://www.LaTeXTemplates.com
%
% Author:
% Vel (vel@latextemplates.com)
%
% License:
% CC BY-NC-SA 4.0 (https://creativecommons.org/licenses/by-nc-sa/4.0/)
%
%%%%%%%%%%%%%%%%%%%%%%%%%%%%%%%%%%%%%%%%%

%----------------------------------------------------------------------------------------
%	PACKAGES AND OTHER DOCUMENT CONFIGURATIONS
%----------------------------------------------------------------------------------------

\documentclass[
	11pt, % Set the default font size, options include: 8pt, 9pt, 10pt, 11pt, 12pt, 14pt, 17pt, 20pt
	%t, % Uncomment to vertically align all slide content to the top of the slide, rather than the default centered
	aspectratio=169, % Uncomment to set the aspect ratio to a 16:9 ratio which matches the aspect ratio of 1080p and 4K screens and projectors
]{beamer}

\graphicspath{{Images/}{./}} % Specifies where to look for included images (trailing slash required)

\usepackage{booktabs} % Allows the use of \toprule, \midrule and \bottomrule for better rules in tables

\usepackage[spanish]{babel}

\usepackage{xcolor}   % for \textcolor
\usepackage{listings} % Package to type bash commands
\lstset{
	basicstyle=\ttfamily,
	columns=fullflexible,
	frame=single,
	showstringspaces=false,
	breaklines=true,
	postbreak=\mbox{\textcolor{red}{$\hookrightarrow$}\space},
	commentstyle=\color{gray},
}

%----------------------------------------------------------------------------------------
%	SELECT LAYOUT THEME
%----------------------------------------------------------------------------------------

% Beamer comes with a number of default layout themes which change the colors and layouts of slides. Below is a list of all themes available, uncomment each in turn to see what they look like.

%\usetheme{default}
%\usetheme{AnnArbor}
%\usetheme{Antibes}
%\usetheme{Bergen}
%\usetheme{Berkeley}
%\usetheme{Berlin}
%\usetheme{Boadilla}
%\usetheme{CambridgeUS}
%\usetheme{Copenhagen}
%\usetheme{Darmstadt}
%\usetheme{Dresden}
%\usetheme{Frankfurt}
%\usetheme{Goettingen}
%\usetheme{Hannover}
%\usetheme{Ilmenau}
%\usetheme{JuanLesPins}
%\usetheme{Luebeck}
%\usetheme{Madrid}
%\usetheme{Malmoe}
%\usetheme{Marburg}
%\usetheme{Montpellier}
%\usetheme{PaloAlto}
\usetheme{Pittsburgh}
%\usetheme{Rochester}
%\usetheme{Singapore}
%\usetheme{Szeged}
%\usetheme{Warsaw}

%----------------------------------------------------------------------------------------
%	SELECT COLOR THEME
%----------------------------------------------------------------------------------------

% Beamer comes with a number of color themes that can be applied to any layout theme to change its colors. Uncomment each of these in turn to see how they change the colors of your selected layout theme.

%\usecolortheme{albatross}
%\usecolortheme{beaver}
%\usecolortheme{beetle}
%\usecolortheme{crane}
%\usecolortheme{dolphin}
%\usecolortheme{dove}
%\usecolortheme{fly}
%\usecolortheme{lily}
%\usecolortheme{monarca}
%\usecolortheme{seagull}
%\usecolortheme{seahorse}
%\usecolortheme{spruce}
%\usecolortheme{whale}
%\usecolortheme{wolverine}

%----------------------------------------------------------------------------------------
%	SELECT FONT THEME & FONTS
%----------------------------------------------------------------------------------------

% Beamer comes with several font themes to easily change the fonts used in various parts of the presentation. Review the comments beside each one to decide if you would like to use it. Note that additional options can be specified for several of these font themes, consult the beamer documentation for more information.

%\usefonttheme{default} % Typeset using the default sans serif font
%\usefonttheme{serif} % Typeset using the default serif font (make sure a sans font isn't being set as the default font if you use this option!)
\usefonttheme{structurebold} % Typeset important structure text (titles, headlines, footlines, sidebar, etc) in bold
%\usefonttheme{structureitalicserif} % Typeset important structure text (titles, headlines, footlines, sidebar, etc) in italic serif
%\usefonttheme{structuresmallcapsserif} % Typeset important structure text (titles, headlines, footlines, sidebar, etc) in small caps serif

%------------------------------------------------

%\usepackage{mathptmx} % Use the Times font for serif text
\usepackage{palatino} % Use the Palatino font for serif text

%\usepackage{helvet} % Use the Helvetica font for sans serif text
\usepackage[default]{opensans} % Use the Open Sans font for sans serif text
%\usepackage[default]{FiraSans} % Use the Fira Sans font for sans serif text
%\usepackage[default]{lato} % Use the Lato font for sans serif text

%----------------------------------------------------------------------------------------
%	SELECT INNER THEME
%----------------------------------------------------------------------------------------

% Inner themes change the styling of internal slide elements, for example: bullet points, blocks, bibliography entries, title pages, theorems, etc. Uncomment each theme in turn to see what changes it makes to your presentation.

%\useinnertheme{default}
\useinnertheme{circles}
%\useinnertheme{rectangles}
%\useinnertheme{rounded}
%\useinnertheme{inmargin}

%----------------------------------------------------------------------------------------
%	SELECT OUTER THEME
%----------------------------------------------------------------------------------------

% Outer themes change the overall layout of slides, such as: header and footer lines, sidebars and slide titles. Uncomment each theme in turn to see what changes it makes to your presentation.

%\useoutertheme{default}
%\useoutertheme{infolines}
%\useoutertheme{miniframes}
\useoutertheme{smoothbars}
%\useoutertheme{sidebar}
%\useoutertheme{split}
%\useoutertheme{shadow}
%\useoutertheme{tree}
%\useoutertheme{smoothtree}

%\setbeamertemplate{footline} % Uncomment this line to remove the footer line in all slides
\setbeamertemplate{footline}[page number] % Uncomment this line to replace the footer line in all slides with a simple slide count

\setbeamertemplate{navigation symbols}{} % Uncomment this line to remove the navigation symbols from the bottom of all slides

%----------------------------------------------------------------------------------------
%	PRESENTATION INFORMATION
%----------------------------------------------------------------------------------------

\title[]{Grimoirebots} % The short title in the optional parameter appears at the bottom of every slide, the full title in the main parameter is only on the title page

\subtitle{Sistema SaaS para el análisis de proyectos software} % Presentation subtitle, remove this command if a subtitle isn't required

\author[]{Sergio Merino Hernández} % Presenter name(s), the optional parameter can contain a shortened version to appear on the bottom of every slide, while the main parameter will appear on the title slide

\institute[]{\includegraphics[height=1cm]{urjc-logo.pdf} \\ \smallskip Máster Universitario en Ingeniería de Telecomunicación} % Your institution, the optional parameter can be used for the institution shorthand and will appear on the bottom of every slide after author names, while the required parameter is used on the title slide and can include your email address or additional information on separate lines

\date[]{\today} % Presentation date or conference/meeting name, the optional parameter can contain a shortened version to appear on the bottom of every slide, while the required parameter value is output to the title slide

%----------------------------------------------------------------------------------------

\begin{document}

%----------------------------------------------------------------------------------------
%	TITLE SLIDE
%----------------------------------------------------------------------------------------

\begin{frame}[plain]
	\titlepage % Output the title slide, automatically created using the text entered in the PRESENTATION INFORMATION block above
\end{frame}

%----------------------------------------------------------------------------------------
%	TABLE OF CONTENTS SLIDE
%----------------------------------------------------------------------------------------

% The table of contents outputs the sections and subsections that appear in your presentation, specified with the standard \section and \subsection commands. You may either display all sections and subsections on one slide with \tableofcontents, or display each section at a time on subsequent slides with \tableofcontents[pausesections]. The latter is useful if you want to step through each section and mention what you will discuss.

\begin{frame}
	\frametitle{Presentation Overview} % Slide title, remove this command for no title

	\tableofcontents % Output the table of contents (all sections on one slide)
	%\tableofcontents[pausesections] % Output the table of contents (break sections up across separate slides)
\end{frame}

%----------------------------------------------------------------------------------------
%	PRESENTATION BODY SLIDES
%----------------------------------------------------------------------------------------

\section{Introducción}

\begin{frame}
	\frametitle{Introducción}

	\begin{figure}
		\includegraphics[width=0.7\linewidth]{intro.jpg}
	\end{figure}
\end{frame}

%------------------------------------------------

\begin{frame}
	\frametitle{GrimoireLab}

	\begin{figure}
		\includegraphics[width=0.8\linewidth]{grimoirelab-schema.png}
	\end{figure}
\end{frame}

%------------------------------------------------

\subsection[]{Motivación}

\begin{frame}
	\frametitle{Motivación}

	\begin{figure}
		\includegraphics[width=0.3\linewidth]{cauldron-logo.png}
	\end{figure}
\end{frame}

%------------------------------------------------

\subsection[]{Objetivos}

\begin{frame}
	\frametitle{Objetivos}

	Desarrollo de un \textbf{servicio web} similar a Cauldron, que permita el análisis de ecosistemas de desarrollo de \emph{software}, modificando algunos de sus principios básicos.

	\bigskip % Vertical whitespace

	Subobjetivos:

	\begin{itemize}
		\item Desarrollar un sistema que genere informes inalterables
		\item Desarrollar múltiples componentes \emph{software} independientes que colaboren entre ellos para dar un servicio al usuario
		\item Simplificar el uso de GrimoireLab mediante la ejecución de su contenedor Docker original
		\item Realizar varias mejoras menores respecto a Cauldron
	\end{itemize}
\end{frame}

%------------------------------------------------

\section{Desarrollo del Proyecto}

%------------------------------------------------

\subsection[]{Análisis de Cauldron}

\begin{frame}
	\frametitle{Análisis de Cauldron}

	\begin{columns}[c] % The "c" option specifies centered vertical alignment while the "t" option is used for top vertical alignment
		\begin{column}{0.45\textwidth} % Left column width
			\textbf{Características}
			\begin{itemize}
				\item Facilitar el uso de GrimoireLab
				\item Django + Elasticsearch
				\item Git, GitHub, Meetup, etc
				\item Informes alterables
				\item Sistema de \emph{workers}
				\item Métricas y gráficas
			\end{itemize}
		\end{column}
		\begin{column}{0.55\textwidth} % Right column width
			\begin{figure}
				\includegraphics[width=\linewidth]{cauldron_tools.pdf}
			\end{figure}
		\end{column}
	\end{columns}
\end{frame}

%------------------------------------------------

\begin{frame}
	\frametitle{Análisis de Cauldron}

	\begin{columns}[c] % The "c" option specifies centered vertical alignment while the "t" option is used for top vertical alignment
		\begin{column}{0.45\textwidth} % Left column width
			\textbf{Puntos de mejora}
			\begin{itemize}
				\item Múltiples iteraciones en los modelos
				\item \emph{Cauldron Pool Scheduler}
				\item Manejo de gran cantidad de datos
				\item Difícil modificación de gráficas
			\end{itemize}
		\end{column}
		\begin{column}{0.55\textwidth} % Right column width
			\begin{figure}
				\includegraphics[width=\linewidth]{cauldron_improvements.pdf}
			\end{figure}
		\end{column}
	\end{columns}
\end{frame}

%------------------------------------------------

\subsection[Arquitectura del software]{Arquitectura del software}

\begin{frame}
	\frametitle{Definición y funcionalidades}

	\textbf{Grimoirebots} es un complejo sistema de análisis de repositorios git que, de forma automática, obtiene y genera datos sobre el desarrollo de \emph{software}.

	\bigskip % Vertical whitespace

	\textbf{Principales características}
	\begin{itemize}
		\item API
		\item OpenSearch Dashboards
		\item Cliente
		\item Herramientas de despliegue
	\end{itemize}
\end{frame}

%------------------------------------------------

\begin{frame}
	\frametitle{Modelos de datos}

	\begin{figure}
		\includegraphics[width=0.8\linewidth]{grimoirebots_models.pdf}
	\end{figure}
\end{frame}

%------------------------------------------------

\begin{frame}
	\frametitle{Descripción de la API}

	\begin{columns}[c] % The "c" option specifies centered vertical alignment while the "t" option is used for top vertical alignment
		\begin{column}{0.4\textwidth} % Left column width
			\begin{figure}
				\includegraphics[width=0.7\linewidth]{order_model.png}
			\end{figure}
		\end{column}
		\begin{column}{0.6\textwidth} % Right column width
			\begin{figure}
				\includegraphics[width=\linewidth]{grimoirebots_api_orders.png}
			\end{figure}
		\end{column}
	\end{columns}
\end{frame}

%------------------------------------------------

\begin{frame}
	\frametitle{Descripción de la API}

	\begin{columns}[c] % The "c" option specifies centered vertical alignment while the "t" option is used for top vertical alignment
		\begin{column}{0.4\textwidth} % Left column width
			\begin{figure}
				\includegraphics[width=0.7\linewidth]{report_model.png}
			\end{figure}
		\end{column}
		\begin{column}{0.6\textwidth} % Right column width
			\begin{figure}
				\includegraphics[width=\linewidth]{grimoirebots_api_reports.png}
			\end{figure}
		\end{column}
	\end{columns}
\end{frame}

%------------------------------------------------

\begin{frame}
	\frametitle{Flujo de ejecución del usuario}

	\begin{figure}
		\includegraphics[width=0.8\linewidth]{grimoirebots_frontend.pdf}
	\end{figure}
\end{frame}

%------------------------------------------------

\begin{frame}
	\frametitle{Flujo de ejecución del cliente}

	\begin{figure}
		\includegraphics[width=0.55\linewidth]{grimoirebots_backend.pdf}
	\end{figure}
\end{frame}

%------------------------------------------------

\section{Ejemplo de uso}

\begin{frame}
	\centering \Huge \color{beamer@blendedblue}
	\textbf{Ejemplo de uso}
\end{frame}

%------------------------------------------------

\section{Conclusiones y trabajos futuros}

%------------------------------------------------

\subsection[]{Conclusiones}

\begin{frame}
	\frametitle{Conclusiones}

	\begin{itemize}
		\item Sistema de análisis para repositorios \emph{software}
		\item Componentes independientes
		\item Simplificación del uso de GrimoireLab
		\item \alert{No logrado}: mejorar las prestaciones de Cauldron
	\end{itemize}
\end{frame}

%------------------------------------------------

\subsection[]{Líneas de desarrollo futuras}

\begin{frame}
	\frametitle{Líneas de desarrollo futuras}

	\begin{columns}[c] % The "c" option specifies centered vertical alignment while the "t" option is used for top vertical alignment
		\begin{column}{0.45\textwidth} % Left column width
			\textbf{Trabajos futuros}
			\begin{itemize}
				\item Sistema de colas
				\item Creación de usuarios
				\item Plataformas Cloud
				\item Interfaz de usuario
				\item CI / CD
			\end{itemize}
		\end{column}
		\begin{column}{0.5\textwidth} % Right column width
			\textbf{Código utilizado}
			\begin{itemize}
				\item \url{github.com/merinhunter/grimoirebots}
				\item \url{github.com/merinhunter/grimoirebots-client}
				\item \url{github.com/merinhunter/grimoirebots-deployment}
			\end{itemize}
		\end{column}
	\end{columns}
\end{frame}

%----------------------------------------------------------------------------------------
%	CLOSING SLIDE
%----------------------------------------------------------------------------------------

\begin{frame}[plain] % The optional argument 'plain' hides the headline and footline
	\begin{center}
		{\Huge The End}

		\bigskip\bigskip % Vertical whitespace

		{\LARGE ¿Alguna pregunta?}
	\end{center}
\end{frame}

%----------------------------------------------------------------------------------------

\end{document}
